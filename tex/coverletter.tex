%        File: coverletter.tex
%     Created: Tue May 31 08:00 PM 2011 P
% Last Change: Tue May 31 08:00 PM 2011 P
%
\documentclass[a4paper]{letter}
\begin{document}
\address{Carl Boettiger \\
Department of Ecology and Evolution \\
University of California \\
One Shields Avenue, \\
Davis CA 95616\\
Fax: (530) 752-3350\\
Carl Boettiger: cboettig@ucdavis.edu  \\
Alan Hastings: amhastings@ucdavis.edu}
\telephone{(530) 752-8116} 
\signature{Carl B and Alan H}
\begin{letter}{ }
\opening{To the editor,}
Enclosed is a manuscript we are submitting as a Letter to Nature.  
We believe that this paper is ideally suited for publication in Nature as it attacks a very important problem -
detecting critical transitions or “tipping points” – that ranges across many fields.  
This work represents a fundamental advance in this problem by providing the first analysis of when early detection will and will not be reliable.
We discover that prevailing proposals based on correlations in summary statistics are highly unreliable on typical data,
but propose a model-based approach which can substantially reduce the fraction false alarms and missed detections.
By quantifying error rates, identifying in what data detection may be probable,
and providing a more powerful tool for detection, 
we believe we have made key steps towards making early warning of sudden shifts possible.

For Nature’s readers:
The ability to predict sudden changes in system state prior to their occurrence is of importance in many fields, but particularly in ecology and environmental science where anthropogenic activities at multiple scales are threatening many systems.  The possibility of doing so without a specific model of a system has been emphasized in a series of papers based essentially on ideas from bifurcations in dynamical systems.  Our work provides the first statistical approach to this problem that properly takes into account issues that arise from considering a single realization and for the first time accurately quantifies both the probability of false alarms and missing impending transitions.

For the general public:
The ability to predict sudden changes in system prior to their occurrence is of importance in many fields, but particularly in ecology and environmental science.  Any forecasting is not without uncertainty; effective management decisions must quantify risks in order to weight the cost of missing an early warning signal against the cost of a false alarm.  Our work provides the first treatment of this uncertainty, and also demonstrating a novel approach for dramatically reducing the error rates.  

We would like to recommend the following individuals as potential reviewers: Dr. Otso Ovaskainen, Dr. Benjamin Bolker, Dr. Stephen Ellner, Dr. N. Thompson Hobbs, Chris Wikle, Dr. Anthony Ives

\closing{Sincerely,}

\encl{ Text: 1496 words (Microsoft Word format) \\
Figures: 3 items, (jpeg format), layout as single figure, shown in pdf.  \\
Supplementary Information (pdf) Contains: 12 sections, 19 pages, 7 figures. \\
Data and source code: all data and code used in the analysis is provided as an R package, a link to the package is provided in the manuscript and in the supplement.  
}

\end{letter}
\end{document}


