%\documentclass[authoryear,5p]{elsarticle}
\documentclass[authoryear,review,12pt]{elsarticle}
\bibliographystyle{elsarticle-harv}
\usepackage{graphicx}
\usepackage{amsmath,amsfonts}
\usepackage{lineno}
\linenumbers
\usepackage{float}
\usepackage{subfigure}
\usepackage[pdftex]{color}
\definecolor{darkblue}{rgb}{0,0,0.5}
\definecolor{darkgreen}{rgb}{0,0.5,0}
\usepackage[pdftex, colorlinks]{hyperref}
\textwidth 6.75in
\oddsidemargin -0.15in
\evensidemargin -0.15in
\textheight 9in
\topmargin -0.5in
\newcommand{\ud}{\mathrm{d}}
\newcommand{\E}{\mathrm{E}}
\newcommand{\C}{\mathrm{Cov}}
\newcommand{\V}{\mathrm{Var}}
\newcommand{\cb}[1]{{\it \color{darkgreen} (#1)}}

%% Redefines the elsarticle footer
\makeatletter
\def\ps@pprintTitle{
\let\@oddhead\@empty
\let\@evenhead\@empty
\def\@oddfoot{\it \hfill\today}
\let\@evenfoot\@oddfoot}
\makeatother



\journal{\tiny } 
\begin{document}
\begin{frontmatter}
  \title{Early Warning Signals and the Prosecutor's Fallacy}
  \author[cpb]{Carl Boettiger\corref{cor1}}
  \ead{cboettig@ucdavis.edu}
  \author[esp]{Alan Hastings}
  %\author[info]{}
  %\author[davis]{}
  \cortext[cor1]{Corresponding author.}
  \address[cpb]{Center for Population Biology, 1 Shields Avenue, University of California, Davis, CA, 95616 United States.}
  \address[esp]{Department of Environmental Science and Policy, University of California, Davis} 

 % \address[info]{ \\ 
  %              }

  \begin{abstract}

  Early warning signals have been proposed to forecast the possibility of a 
  critical transition, such as the eutriphication of a lake, desertification
  of savanna, the collapse of a coral reef, the end of a glacial period, or 
  the onset of epilepsy in the brain.  While such transitions may be difficult
  to study in the laboratory setting and unethical to introduce on the a larger
  natural scales where they may occur, it has often been observed that history
  has already performed the experiment for us, time and again.  Here we point 
  out a critical difference between selecting systems for study based on the 
  fact that we have observed a critical transition and those systems for which
  we wish to forecast the approach of a transition. This difference arising
  from conditional probabilities is often known as the the Prosecutor's 
  Fallacy.  We illustrate through simulation how examples of systems that have 
  experienced transitions purely by chance show patterns associated with 
  early warning signals more often than we might expect in systems that have not
  been conditionally selected.  We further highlight how a model-based approach
  is less subject to this bias.  In using historical examples of collapse to 
  test and validate methods for early warning signals, greater care must be taken
  to avoid this fallacy.
 
  \end{abstract}

  \begin{keyword}
early warning signals \sep tipping point \sep alternative stable states \sep likelihood methods 
   \end{keyword}
 \end{frontmatter}

\section{Introduction}

\subsection*{Catastrophic Transitions}

Catastrophic transitions or tipping points, where a complex system 
shifts suddenly from one state to another, have been implicated in
a wide array of ecological and global climate systems such as lake
ecosystems~\citep{Carpenter2011}, coral reefs~\citep{Mumby2007},
desertification~\citep{Kefi2007}, fisheries~\citep{Berkes2006}, and 
tropical forests~\citep{Hirota2011}.  Recent research has begun to
identify statistical patterns commonly associated with these sudden
catastrophic transitions which could be used as an \emph{early warning sign}
to identify an approaching tipping point, buying precious time 
to react and avert them~\citep{Scheffer2009, Lenton2011}.  

An array of statistical patterns associated with tipping point phenomena has
been suggested for the detection of early warning signals associated with
such sudden transitions.  Two of the most commonly used are a pattern of
increasing variance~\citep{Carpenter2006} and a pattern of increasing 
autocorrelation~\citep{vanNes2007}, which have been tested in both experimental
manipulation~\citep{Drake2010, Carpenter2011, Veraart2011} and historical 
observations~\citep{Livina2007,Dakos2008,Lenton2012,Ditlevsen2010,Guttal2008,Thompson2010}


\subsection*{Testing patterns on historical data}

Historical examples of sudden transitions taken from the paleoclimate record
provide an important way to test and evaluate potential leading indicator 
methods, and have been widely used for this purpose 
\citep{Livina2007,Dakos2008,Lenton2012,Ditlevsen2010,Guttal2008,Thompson2010}.
Similarly, it has been suggested that data gathered from ecological systems such
as lakes that were monitored before the collapsed, or grasslands subjected to
overgrazing, could contain data that could help reveal when similar systems 
are approaching a tipping point~\citet{Inman2011}.  This is a silver lining to
the clouds of past catastrophes -- we may be able to learn from our mistakes. 

Yet in testing methods for early warning signals against historical examples of
transitions, we must be careful to avoid statistical mistakes that arise when
we select examples conditional on having exhibited a sudden transition.  Such
sudden transitions can occur for many reasons that do not involve the 
bifurcation processes on which most early warning indicators are predicated
~\citep{Hastings2010}, including purely noise-induced transitions for which 
early warning signs will not exist~\citep{Ditlevsen2010, Lenton2011}. Treating 
such conditionally selected historical examples as if they were randomly 
sampled representatives of systems that have experienced a saddle-node 
bifurcation is an example of the statistical error commonly known as the
\emph{Prosecutor's Fallacy.}

\subsection*{The Prosecutor's Fallacy}


In a criminal trial, the prosecution may introduce associative evidence such as 
blood testing to establish a match between the perpetrator at the scene of the
crime and a suspect on the defense stand.  Statistical testimony may accompany 
such evidence reflecting the incidence rate of the matching characteristic -- 
for instance, the frequency of individuals in the population with the matching
blood type.  Such arguments may be misleading when they fail to reflect the 
probability conditional on the manner in which the suspect was identified in 
the first place.  If the suspect has been selected partly on the basis of 
belonging to demographic in which the characteristic is much more common than
in the population as a whole, these odds can be misleading.  Failing to recognize 
these conditional probabilities leads to the Prosecutor's Fallacy -- the 
argument that the suspect is the perpetrator on the basis of the statistical 
probability of a match without correctly identifying the baseline probability 
for such a match \citep{Thompson1987}.


\section{Methods and Results}
We approach the problem by simulating a system that cannot undergo the saddle
node bifurcation that underlies the usual early warning signal indicators.  
This simulation approach is the only way to determine whether examining 
historical events is a good way to test the utility of these indicators.
We simulated 1000 replicates of an individual-based birth-death process
with an Allee threshold.  Above the Allee threshold the population returns
to a positive equilibrium size.  Below the threshold the population decreases
to zero. The simulation uses the Gillespie algorithm to provide an exact 
implementation of the master equation of birth-death process,

\begin{align}
  \frac{dP(n,t)}{dt} &= b_{n-1} P(n-1,t) + d_{n+1}P(n+1,t) - (b_n+d_n) P(n,t)  \label{master} \\
    b_n &= \frac{e K n^2}{n^2 + h^2} \\
    d_n &= e n + a
\end{align}

Though this system can be forced through a saddle-node bifurcation by
increasing values of either $e$ or $a$, in these simulations all parameters
are held constant and no early warning signal is anticipated.
The simulation starts from the positive equilibrium population size.
Though the chance of a transition across the Allee threshold in any 
given time step are small, given enough time this system will eventually
experience such a rare event driving the population extinct.  We evolved
each replicate 50000 time steps, sampling the system every 50 time steps.  
In this time window 266 of the 1000 replicates experience population collapse.  
Nine of these replicates can be seen in Figure~\ref{fig:replicate_crashes},
as an illustration. 

  \begin{figure}[H]
    \begin{center}
      \includegraphics{replicate_crashes.pdf}
    \end{center}
    \caption{Example trajectories for nine of the 266 replicates that 
             experience a critical transition by chance during the course
             of the simulation.}
    \label{fig:replicate_crashes}
  \end{figure}


We selected replicates conditional on having collapsed in the simulations.
We then selected a window around each system that ended just before the
collapse, while the population values were still above the Allee threshold.
For each replicate, we calculated the most common warning signals, variance
and autocorrelation~\citep[\emph{e.g.}][]{Carpenter2006,Dakos2008,Scheffer2009}, 
around a moving window equal to half the length of that time series.  

To determine whether or not the variance or autocorrelation increased 
over time in these 266 replicates, we computed Kendall's $\tau$ for each of the
warning signal pattern~\citep{Dakos2008, Dakos2011}.  Positive values of $\tau$ 
indicate an increasing trend in the warning signal.  The distribution of $\tau$ 
values observed across these replicates is shown in Figure~\ref{fig:indicator}.
More positive values indicate stronger trends. Assessing the statistical significance
of a given value requires an appropriate null model, which can be difficult 
to do~\citep{Dakos2008}. As these examples have been selected from a system
that is not approaching a bifurcation and therefore should not exhibit any 
early warning sign. %Even the very strong values of $\tau$ should be considered 
%as arising from the null model. 


\begin{figure}[H]
  \begin{center}
    \includegraphics{indicators.pdf}
  \end{center}
  \caption{The distribution of the correlation statistic $\tau$ for two early warning indicators (variance, autocorrelation) on replicates conditionally selected for having collapsed by chance in simulations. Positive values of $\tau$ correspond to a pattern of an indicator increasing with time; typically taken as evidence that a system is approaching a critical transition.  In these simulations, the pattern arises instead from the Prosecutor's fallacy of conditional selection.}
  \label{fig:indicator}
\end{figure}

For each of these replicates we also take a model-based approach, estimating 
parameters for an approximate linear model of the system approaching a
saddle node bifurcation.


\begin{equation}
  dX = \sqrt{ r_t } (\phi(r_t) - X_t) dt +
  \sigma\sqrt{\phi(r_t) } d B_t \label{LSN}
\end{equation}

where \( \phi(r_t) = \sqrt{r_t} +\theta \) is the equilibrium position.  
This model emerges from the canonical form of the saddle-node bifurcation,

\begin{equation}
  \frac{dX}{dt} = r - x^2
\end{equation}

Where we introduce an explicit parameter $\theta$ to account for rescaling
to the units of the data,

\begin{equation}
   \frac{dX}{dt} = r - (x-\theta)^2
\end{equation}

and then consider the corresponding stochastic process where the noise is
proportional to the square root of the mean, as is the case for demographic
noise processes,

\begin{equation}
  \frac{dX}{dt} = r - (x-\theta)^2 + \sqrt{ \beta (r + (x-\theta)^2 } dB_t
\end{equation}

$r$ remains the bifurcation parameter which we imagine changing with time,
$r_t$, then linearizing the above model in neighborhood of
\(\hat x = \sqrt{r_t} +\theta =: \phi(r_t)\), we recover Equation~\eqref{LSN}.
These models can be both be fit to the data by maximum likelihood estimation
based on the equation for the evolution of their moments, as
in~\citet{Boettiger2012b}.  

Though this approximation is more general than the original Markov process,
it is also easier to estimate reliably from the data since it describes only 
the process near the equilibrium where the data is available.  

We assume that $r_t$ is varying at constant rate, according to the model

\begin{equation}
 r_t = m t + r_0 
\end{equation}

and estimate $m$, $r_0$, $\theta$, and $\beta$ by maximum likelihood on each
of the simulated datasets.  Estimates of $m < 0 $ are expected in systems 
approaching a bifurcation, while for stable systems $m$ should be approximately zero.
Of the 266 simulations, only six had estimates with machine precision different
from zero, the greatest magnitude among these was $-7.1 \times 10^{-12}$,
which would still correspond to a negligible change in $r_0$ over the 50,000 time steps
of the simulation.  Consequently, the model-based estimation shows no 
evidence of bias on data that has been selected conditional on collapse.   

 \section{Discussion}

We have shown that selecting datasets conditional on having experienced sudden 
noise-induced transitions can lead to patterns that are more likely to resemble
early warning signals typically associated with critical slowing down before 
a bifurcation event.  Purely noise-induced transitions are distinctly separate 
mechanism from critical slowing-down phenomena, in which the transition occurs
as the result of a single or series of extremely unlikely events rather than
a slow but regular degradation of the system stability.  

\cb{The discussion that follows here is aimed at giving some intuition for our results, but comes off as a bit more technically focused than the section above.  Should it be moved into an earlier section, or do we simply need a summarizing paragraph at the end?}

It seems tempting to argue that the bias towards positive detection in historical
examples is in fact valuable -- each of these systems did indeed collapse, so the fact that they show increased 
probability to exhibit warning signals could be taken as a successful detection.
Unfortunately, this reasoning is flawed. At the moment the forecast is made,
these systems are no more likely to crash than their counterparts from the sample
that did not collapse, just as a fair coin is still a fair coin after a string of
consecutive ``heads.''  Taking a closer look at the patterns involved we can
begin to understand why common indicators such as autocorrelation and variance
can be misleading.

\cb{ Delete this paragraph?}
The system simulated in our examples is governed by a birth-death process --
a discrete Markov process that can in any instant only change by a single 
birth or single death.  At each moment in time, there is some probability 
that either event will occur next.  The system near its stable equilibrium
can be well approximated as a Gaussian process, where consecutive random 
draws buffet it above and below the stable point. While these random draws
vary in magnitude, recall that this is simply an approximation to the 
underlying process where events are all of the same magnitude but vary
in their frequency following a state-dependent Poisson process.  This 
means that what looks like a large random draw from the Gaussian distribution
really corresponds to a series of more rapid events with some excess of 
birth or deaths.  

The farther the system gets from that stable point (while still 
remaining in the local region), the more likely it is to draw a random step 
that returns it towards the stable point, like a spring that is stretched 
further and further, but there is some probability that it will move further
still. Systems that manage to extend this random walk far enough to cross the
tipping point must do so rather quickly -- with a string of events moving 
closer and closer to the edge.  This pattern, clearly visible before the 
crashes in each of the examples in Figure 1, produces a string of observations
that appear more highly autocorrelated (if we are sampling the system
frequently enough to catch the excursion at all) than we observe in the rest
of the fluctuations around the equilibrium.  Yet this autocorrelation comes from
a chance trajectory moving quickly \emph{away} from the stable state, not from
the critical slowing down pattern in the return times to the stable state
which precede a saddle-node bifurcation and motivate the early warning signal.



This longer than expected excursion results in a higher than expected variance
in that window as well. Both variance and autocorrelation are calculated using
a moving window over the time-series, which allows the method to pick out a 
pattern of change as the window moves along the sequence. If this chance excursion 
that precedes the crash happens to fill a significant part of the moving window,
the resulting pattern will tend to show an increase in autocorrelation or variance.
If the chance excursion is relatively rapid compared to the frequency at which
the system is observed (spacing of the data) or the width of the moving window, 
the excursion may not significantly alter the general pattern.  In this way, 
some of the events in which a crash is observed will appear to present these
statistical patterns of increased variance or autocorrelation without being
harbingers of approaching critical transitions.  

Because we have selected our data by conditioning
on the fact that they have experienced such a crash, these excursions appear
more frequently than we would expect by chance in the observations of a given
stable system over some finite time window.  That biased selection is the
heart of the Prosecutor's Fallacy.  In these cases, the presence of increasing
autocorrelation or increasing variance does not represent the critical slowing
down process.

The problem we highlight ultimately stems from the difficulty of having only a
single realization with which to examine a complex problem.  The only way to deal
with this problem is through replication, as can be done in an experimental system
in laboratory manipulations such as~\citet{Drake2010, Veraart2011} and at the 
scale of whole lake ecosystems in~\citet{Carpenter2011}.  Historical data will 
not suffice, more such replicated experiments are needed to test these patterns.  


 \section{Acknowledgements}
This research was supported by funding from NSF Grant EF 0742674 to AH
and a Computational Sciences Graduate Fellowship from the Department of Energy grant DE-FG02-97ER25308 to CB. 
 \section{References}%bibliography
 \bibliography{fallacy}
 \end{document}



