%\documentclass[authoryear,5p]{elsarticle}
\documentclass[authoryear,preprint,11pt]{elsarticle}
\bibliographystyle{elsarticle-harv}
\usepackage{graphicx}
\usepackage{amsmath,amsfonts}
\usepackage{lineno}
\linenumbers
\usepackage{subfigure}
\usepackage[pdftex]{color}
\definecolor{darkblue}{rgb}{0,0,0.5}
\definecolor{darkgreen}{rgb}{0,0.5,0}
%\usepackage[pdftex, colorlinks, citecolor=darkblue,linkcolor=darkgreen]{hyperref}
\usepackage[pdftex, colorlinks]{hyperref}
\textwidth 6.75in
\oddsidemargin -0.15in
\evensidemargin -0.15in
\textheight 9in
\topmargin -0.5in
\newcommand{\ud}{\mathrm{d}}
\newcommand{\E}{\mathrm{E}}
\newcommand{\C}{\mathrm{Cov}}
\newcommand{\V}{\mathrm{Var}}

\newcommand{\cb}[1]{{\it \color{darkgreen} (#1)}}



\journal{\tiny } 
\begin{document}
\begin{frontmatter}
  \title{Early Warning Signals and the Prosecutor's Fallacy}
  \author[cpb]{Carl Boettiger\corref{cor1}}
  \ead{cboettig@ucdavis.edu}
  \author[esp]{Alan Hastings}
  %\author[info]{}
  %\author[davis]{}
  \cortext[cor1]{Corresponding author.}
  \address[cpb]{Center for Population Biology, 1 Shields Avenue, University of California, Davis, CA, 95616 United States.}
  \address[esp]{Department of Environmental Science and Policy, University of California, Davis} 

 % \address[info]{ \\ 
  %              }

  \begin{abstract}

  Early warning signals have been proposed to forecast the possibility of a 
  critical transition, such as the eutriphication of a lake, desertification
  of savanna, the collapse of a coral reef, the end of a glacial period, or 
  the onset of epilepsy in the brain.  While such transitions may be difficult
  to study in the laboratory setting and unethical to introduce on the a larger
  natural scales where they may occur, it has often been observed that history
  has already performed the experiment for us, time and again.  Here we point 
  out a critical difference between selecting systems for study based on the 
  fact that we have observed a critical transition and those systems for which
  we wish to forecast the approach of a transition. This difference arising
  from conditional probabilities is often known as the the Prosecutor's 
  Fallacy.  As we pursue this important question of early warning signals, 
  we must be careful to avoid this trap.  

  \end{abstract}

  \begin{keyword}
early warning signals \sep tipping point \sep alternative stable states \sep likelihood methods 
   \end{keyword}
 \end{frontmatter}
 \section{References}%bibliography
 \bibliography{boettiger}
 \end{document}



